
\csection{Введение}

В современном мире сбор и анализ данных играют ключевую роль в принятии решений во многих сферах деятельности. 
Благодаря развитию технологий, стало возможным эффективно использовать огромные объемы данных для улучшения 
различных процессов и повышения эффективности работы. В особенности, методы компьютерного зрения и искусственного 
интеллекта открывают новые возможности для автоматизации и оптимизации задач, которые ранее выполнялись вручную.

С развитием технологий обработки данных и машинного обучения, особенно нейронных сетей, стало возможным решать 
сложные задачи, такие как распознавание образов, анализ поведения, предсказания исходов каких-любо событи и так далее. 
Эти технологии находят широкое применение в самых разных областях, от маркетинга и розничной 
торговли до здравоохранения и транспорта.

Современные маркетологи сталкиваются с необходимостью глубоко понимать интересы и предпочтения своих клиентов, чтобы создавать эффективные стратегии и кампании. Проект по автоматизации процесса сбора и анализа интересов клиентов автостоянки предоставляет маркетологам мощный инструмент для достижения этой цели.

Благодаря полученным данным, маркетологи смогут более точно сегментировать аудиторию, разрабатывать персонализированные предложения и эффективно проводить рекламные кампании. Это приведет к улучшению взаимодействия с клиентами, увеличению их лояльности и, в конечном итоге, к росту продаж и прибыли компании.

Исходя из выше сказанного, было принято решение о создании системы, которая смогла бы автоматизировать процесс сбора и 
анализа интересов клиентов автостоянки, для того чтобы предоставить пользователям информацию о посетителях.

\pagebreak
